\documentclass[11pt]{article}
\usepackage{amsmath,amssymb}
\usepackage{geometry}
\geometry{margin=1in}

\title{Numerical Consistency of the Riemann--von Mangoldt Formula\\
and the Finite Validity of Polynomial Corrections}
\author{Author Name}
\date{}

\begin{document}
\maketitle

\begin{abstract}
We examine the numerical consistency of the Riemann--von Mangoldt
formula for the non-trivial zeros of the Riemann zeta function.
Using the first 200 zeros, we introduce a minimal polynomial
correction and analyze the residual structure.
While such corrections effectively absorb local deviations,
extrapolation to higher zeros (501--700) fails, indicating that
the correction is only locally valid.
We conclude that the Riemann--von Mangoldt leading term is the
only universal structure, and that the remaining fluctuations
are structureless.
\end{abstract}

\section{Introduction}
Let $\rho = \tfrac12 + i\gamma_n$ denote the non-trivial zeros of the
Riemann zeta function, ordered by increasing imaginary part.
The counting function for zeros with $0 < \gamma \le T$ is well
approximated by the Riemann--von Mangoldt formula
\[
N_0(T)
= \frac{T}{2\pi}\log\frac{T}{2\pi}
- \frac{T}{2\pi}.
\]
The purpose of this work is to examine, using explicit numerical
data, whether systematic corrections to this leading term exhibit
any universal structure beyond finite ranges.

\section{Method}
Using the first 200 non-trivial zeros $\{\gamma_n\}$, we model the
relation
\[
n = N_0(\gamma_n) + \Delta(\gamma_n),
\]
where $\Delta(\gamma)$ is taken to be a minimal quadratic polynomial
\[
\Delta(\gamma) = A\gamma^2 + B\gamma + C.
\]
The coefficients $A,B,C$ are determined by least-squares fitting.
Residuals are defined as
\[
r_n = n - \bigl(N_0(\gamma_n)+\Delta(\gamma_n)\bigr),
\]
and are analyzed after logarithmic normalization.
To test universality, the same coefficients are applied without
refitting to an independent set of higher zeros, namely
$\gamma_{501}$--$\gamma_{700}$.

\section{Results for Zeros 1--200}
The least-squares fit over the first 200 zeros yields
\[
A = -63.16612305756696,\quad
B = 324.0670957385392,\quad
C = -2891.544542946499.
\]
Within this finite interval, the polynomial correction removes the
systematic deviation from the leading term.
The normalized residuals exhibit an approximately Gaussian
distribution with standard deviation
\[
\mathrm{std} \approx 0.579548658,
\]
and show no statistically significant periodicity or higher-order
structure.

\section{Extrapolation Test: Zeros 501--700}
Applying the same coefficients $A,B,C$ to zeros
$\gamma_{501}$--$\gamma_{700}$ leads to rapidly growing residuals.
A refit over this higher range yields an optimal quadratic
coefficient $A \approx 0$, indicating that no polynomial correction
is required.
Thus, the quadratic correction determined from the first 200 zeros
has no extrapolative validity and represents only a local
approximation.

\section{Discussion}
The extrapolation failure demonstrates that the polynomial
correction does not encode a new universal law.
Instead, it functions as a finite-range error absorber.
The only structure that remains stable across ranges is the
Riemann--von Mangoldt leading term itself; beyond this term,
the residual fluctuations are consistent with structureless noise.
The explicit detection of overfitting confirms the limited scope of
such corrections.

\section{Conclusion}
We conclude that:
\begin{itemize}
\item The Riemann--von Mangoldt leading term is the sole universal
component governing zero counts.
\item Polynomial corrections are valid only locally and have no
predictive power outside their fitting range.
\item After removal of the leading term, residuals exhibit no
deterministic structure.
\end{itemize}
These results fix the numerical scope of admissible corrections and
provide a closed empirical basis for further theoretical
interpretation.

\begin{thebibliography}{1}
\bibitem{Odlyzko}
A.~M.~Odlyzko,
\emph{Tables of zeros of the Riemann zeta function}.
\end{thebibliography}

\end{document}
